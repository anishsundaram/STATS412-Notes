%========================%
%        Preamble        %
%========================%
\documentclass[12pt]{amsart}

    %========================%
%        Packages        %
%========================%

\usepackage[utf8]{inputenc}
%\usepackage{amsmath}    % Included in amsart package
%\usepackage{amsthm}     % 
\usepackage{amssymb}      % 
\usepackage{mathtools}      % Paired Limiter Macros
% \usepackage{mdframed}       % boxes for theorem
\usepackage{enumitem}     % Continuous numbering of lists
\usepackage[hidelinks]{hyperref}
\usepackage{tikz}
\usetikzlibrary{positioning}
\usepackage{blindtext}
\usepackage{graphicx}
\usepackage{float}

%========================% 
%          Title         %
%========================% 
\title{Lecture 11 and 12}
\author{Anish Sundaram}
\date{\today}

%========================% 
%        Theorems        %
%========================% 
\theoremstyle{definition}
\newtheorem{theorem}{Theorem}  % Boxed theorems
\newtheorem{definition}{Definition} % Definitions
\newtheorem{example}{Example}       %
\newtheorem{algorithm}{Algorithm}
\newtheorem*{proof*}{Proof}         % non-numbered
\newtheorem*{remark}{Remark}        %
\numberwithin{equation}{theorem}    % Local equation numbering

\setcounter{tocdepth}{3}      % Show subsubsections in contents

%========================% 
%        Macros          %
%========================% 
\DeclarePairedDelimiter\abs{\lvert}{\rvert}  % Vertical bars
\DeclarePairedDelimiter\norm{\lVert}{\rVert} % Double vertical bars
\newcommand{\drawvec}[1]{                    % matrices on one line
    \begin{bmatrix}
        #1
    \end{bmatrix}
}


% \begin{figure}[H]
%     \centering
%     \includegraphics[width=5in]{global-carbon-cycle.png}
%     \caption{The Global Carbon Cycle}
%     \label{global-carbon-cycle}
% \end{figure}

%========================% 
%         Document       %
%========================% 
\begin{document}

\maketitle

\tableofcontents

\section*{11 Important Families of Continuous Probability Distributions}
\subsection*{11.1 Exponential Distributions}

\begin{definition}
    \textbf{Exponential Distributions}:
    Used to model a process in which events occur continuously and independently at a constant average rate. The formula for these distributions is:
    $$f(x) = \begin{cases}
        \lambda e ^{-\lambda x}  & \text{if } x > 0 \\
        0  & otherwise
  \end{cases} \quad$$
  wherein $\lambda$ is referred to as the rate.
  \begin{itemize}
    \item $\lambda = \frac{1}{X}$
    \item $Mean:E[X] = \frac{1}{\lambda}$
    \item $Variance: \sigma_x^2 = \frac{1}{\lambda^2}$
    \item $Stdev: \sigma_x = \sqrt{\frac{1}{\lambda^2}} = \frac{1}{X\sqrt{n}}$
    \item $CDF: P(X < x) = \begin{cases}
        1-e ^{-\lambda x}  & \text{if } x \geq 0 \\
    \end{cases} \quad$
    \item $CDF: P(X > x) = \begin{cases}
        e ^{-\lambda x}  & \text{if } x \geq 0 \\
    \end{cases} \quad$
\end{itemize}
\end{definition}



\subsection*{11.2 Gamma Distribution}

\begin{definition}
    \textbf{Gamma Distribution}:
    The gamma distribution is a continuous distribution, one of whose purposes is to extend the usefulness of the exponential distribution in modeling waiting times. It involves a certain integral known as the gamma function. 
    Described by the PDF:
    $$f(x) = \begin{cases}
        \frac{\lambda^rx^{r-1}e^{-\lambda x}}{\Gamma(r} & \text{if } x > 0 \\
        0  & otherwise
  \end{cases} \quad$$
  \begin{itemize}
    \item $\lambda = \frac{1}{X}$
    \item $Mean:E[X] = \frac{r}{\lambda}$
    \item $Variance: \sigma_x^2 = \frac{r}{\lambda^2}$
\end{itemize}
\end{definition}

\end{document}