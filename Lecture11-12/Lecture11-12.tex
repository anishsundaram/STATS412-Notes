%========================%
%        Preamble        %
%========================%
\documentclass[12pt]{amsart}

    %========================%
%        Packages        %
%========================%

\usepackage[utf8]{inputenc}
%\usepackage{amsmath}    % Included in amsart package
%\usepackage{amsthm}     % 
\usepackage{amssymb}      % 
\usepackage{mathtools}      % Paired Limiter Macros
% \usepackage{mdframed}       % boxes for theorem
\usepackage{enumitem}     % Continuous numbering of lists
\usepackage[hidelinks]{hyperref}
\usepackage{tikz}
\usetikzlibrary{positioning}
\usepackage{blindtext}
\usepackage{graphicx}
\usepackage{float}

%========================% 
%          Title         %
%========================% 
\title{Lecture 11 and 12}
\author{Anish Sundaram}
\date{\today}

%========================% 
%        Theorems        %
%========================% 
\theoremstyle{definition}
\newtheorem{theorem}{Theorem}  % Boxed theorems
\newtheorem{definition}{Definition} % Definitions
\newtheorem{example}{Example}       %
\newtheorem{algorithm}{Algorithm}
\newtheorem*{proof*}{Proof}         % non-numbered
\newtheorem*{remark}{Remark}        %
\numberwithin{equation}{theorem}    % Local equation numbering

\setcounter{tocdepth}{3}      % Show subsubsections in contents

%========================% 
%        Macros          %
%========================% 
\DeclarePairedDelimiter\abs{\lvert}{\rvert}  % Vertical bars
\DeclarePairedDelimiter\norm{\lVert}{\rVert} % Double vertical bars
\newcommand{\drawvec}[1]{                    % matrices on one line
    \begin{bmatrix}
        #1
    \end{bmatrix}
}


% \begin{figure}[H]
%     \centering
%     \includegraphics[width=5in]{global-carbon-cycle.png}
%     \caption{The Global Carbon Cycle}
%     \label{global-carbon-cycle}
% \end{figure}

%========================% 
%         Document       %
%========================% 
\begin{document}

\maketitle

\tableofcontents

\section*{11 Important Families of Continuous Probability Distributions}
\subsection*{11.1 Exponential Distributions}

\begin{definition}
    \textbf{Exponential Distributions}:
    Used to model a process in which events occur continuously and independently at a constant average rate. The formula for these distributions is:
    $$f(x) = \begin{cases}
        \lambda e ^{-\lambda x}  & \text{if } x > 0 \\
        0  & otherwise
  \end{cases} \quad$$
  wherein $\lambda$ is referred to as the rate.
  \begin{itemize}
    \item $\lambda = \frac{1}{X}$
    \item $Mean:E[X] = \frac{1}{\lambda}$
    \item $Variance: \sigma_x^2 = \frac{1}{\lambda^2}$
    \item $Stdev: \sigma_x = \sqrt{\frac{1}{\lambda^2}} = \frac{1}{X\sqrt{n}}$
    \item $CDF: P(X < x) = \begin{cases}
        1-e ^{-\lambda x}  & \text{if } x \geq 0 \\
    \end{cases} \quad$
    \item $CDF: P(X > x) = \begin{cases}
        e ^{-\lambda x}  & \text{if } x \geq 0 \\
    \end{cases} \quad$
\end{itemize}
\end{definition}



\subsection*{11.2 Gamma Distribution}

\begin{definition}
    \textbf{Gamma Distribution}:
    The gamma distribution is a continuous distribution, one of whose purposes is to extend the usefulness of the exponential distribution in modeling waiting times. It involves a certain integral known as the gamma function. 
    Described by the PDF:
    $$f(x) = \begin{cases}
        \frac{\lambda^rx^{r-1}e^{-\lambda x}}{\Gamma(r} & \text{if } x > 0 \\
        0  & otherwise
  \end{cases} \quad$$
  \begin{itemize}
    \item $\lambda = \frac{1}{X}$
    \item $Mean:E[X] = \frac{r}{\lambda}$
    \item $Variance: \sigma_x^2 = \frac{r}{\lambda^2}$
\end{itemize}
\end{definition}


\section*{12 Simulations.The Normal Distribution and the CLT}

\subsection*{12.1 The Normal Distribution}


\begin{definition}
    \textbf{Normal Distribution}:
    Normal distribution, also known as the Gaussian distribution, is a probability distribution that is symmetric about the mean, showing that data near the mean are more frequent in occurrence than data far from the mean. In graph form, normal distribution will appear as a bell curve. Normal distributions are described by 

    \begin{itemize}
        \item $X = \mu + Z\sigma$
        \item $Mean:\mu= np$
        \item $Standard Deviation: \sigma = \sqrt{npq}$
        \item $Variance: \sigma_x^2 = npql$
        \item $PDF: \frac{1}{\sigma \sqrt{2\pi}}e^{-.5(\frac{X-\mu}{\sigma})^2}$
    \end{itemize}
    \begin{remark}
        The normal distribution is commonly associated with the 68-95-99.7 rule which you can see in the image above. 68 $\%$ of the data is within 1 standard deviation of the mean , 95 $\%$ of the data is within 2 standard deviations of the mean, and 99.7 $\%$ of the data is within 3 standard deviations of the mean
    \end{remark}
\end{definition}

\begin{definition}
    \textbf{Standard Normal Distribution}:
    The ideal form of a normal distribution wherein the mean is 0 and the standard deviation is 1, and the distribution the tables work on. To convert to a standard normal we need a z-score.
\end{definition}


\begin{definition}
    \textbf{Z-score}:
    A method of converting any distribution into one appropirate for using a standard normal Distribution table:
    $$Z = \frac{X-\mu}{\sigma}$$
    Z is a measure of how far a value is from the mean in terms of standard deviations.
\end{definition}


\begin{definition}
    \textbf{Sum of normal Randvars}:
    If the 2 randvars are independent, for the mean and variance of X+Y we have 
    $$\mu = \mu_X + \mu_Y$$
    and $$\sigma^2 = \sigma_X^2 + \sigma_Y^2$$
\end{definition}

\subsection*{12.2 Central Limit Theorem}

\begin{definition}
    \textbf{Central Limit Theorem(CLT)}:
    As $n$ increases, the distribution of the sample mean or sum approaches a normal distribution. In practice, if n$\geq$30 we can normalize the distribution.
\end{definition}


\subsection*{12.3 Simulation of Random Variables}

\begin{definition}
    \textbf{Simulation of Randvar}
\end{definition}


\subsection*{12.4 Monte Carlo Simulation}

\begin{definition}
    \textbf{Monte Carlo Simulation}:
    A method of estimating the value of an unkown quantity unsing inferential statistics.
\end{definition}

\end{document}