%========================%
%        Preamble        %
%========================%
\documentclass[12pt]{amsart}

    %========================%
%        Packages        %
%========================%

\usepackage[utf8]{inputenc}
%\usepackage{amsmath}    % Included in amsart package
%\usepackage{amsthm}     % 
\usepackage{amssymb}      % 
\usepackage{mathtools}      % Paired Limiter Macros
% \usepackage{mdframed}       % boxes for theorem
\usepackage{enumitem}     % Continuous numbering of lists
\usepackage[hidelinks]{hyperref}
\usepackage{tikz}
\usetikzlibrary{positioning}
\usepackage{blindtext}
\usepackage{graphicx}
\usepackage{float}

%========================% 
%          Title         %
%========================% 
\title{Lecture 5 and 6}
\author{Anish Sundaram}
\date{\today}

%========================% 
%        Theorems        %
%========================% 
\theoremstyle{definition}
\newtheorem{theorem}{Theorem}  % Boxed theorems
\newtheorem{definition}{Definition} % Definitions
\newtheorem{example}{Example}       %
\newtheorem{algorithm}{Algorithm}
\newtheorem*{proof*}{Proof}         % non-numbered
\newtheorem*{remark}{Remark}        %
\numberwithin{equation}{theorem}    % Local equation numbering

\setcounter{tocdepth}{3}      % Show subsubsections in contents

%========================% 
%        Macros          %
%========================% 
\DeclarePairedDelimiter\abs{\lvert}{\rvert}  % Vertical bars
\DeclarePairedDelimiter\norm{\lVert}{\rVert} % Double vertical bars
\newcommand{\drawvec}[1]{                    % matrices on one line
    \begin{bmatrix}
        #1
    \end{bmatrix}
}


% \begin{figure}[H]
%     \centering
%     \includegraphics[width=5in]{global-carbon-cycle.png}
%     \caption{The Global Carbon Cycle}
%     \label{global-carbon-cycle}
% \end{figure}

%========================% 
%         Document       %
%========================% 
\begin{document}

\maketitle

\tableofcontents

\section*{5 Random Variables}
\subsection*{5.1 Random Variables}

\begin{definition}
    \textbf{Random Variable($X$)}:
    A variable whose value is the numerical output of a random event.
    There are many forms of Randvars but each is the output of some probability function. 
\end{definition}

\subsection*{5.2 Discrete Random Variables}

\begin{definition}
    \textbf{Discrete Random Variable}:
    A random variable $X$ is said to be discrete if it takes either finitely many or a countable many values.
    \begin{remark}
        Examples: \begin{enumerate}
            \item Experiment: Toss a fair coin $n$ times and let $X$ be the number ofHeads.
            \item Experiment: The number $X$ of accidents on I-95 during next week.
            \item Experiment: Let $X$ be the number of coin-tosses until the first Head comes up
        \end{enumerate}  
    \end{remark}
\end{definition}


\begin{definition}
    \textbf{Continous Random Variable}:
    A random variable $X$ is said to be discrete if it can take any number within a variable.
    \begin{remark}
        Examples include the exact time of an event happening or measurements.
    \end{remark}
\end{definition}

\subsection*{5.2 Probability Mass Function}

\subsection*{5.3 The Bernoulli Distributions}

\begin{definition}
    \textbf{Bernoulli Trial}:
    A defined single trial that is classified as:
    \begin{enumerate}
        \item Binary Outcomes, ie Success or Failure
        \item Independent observations
        \item Fixed probability
    \end{enumerate}
    \begin{remark}
        Some times Bernoulli Randvars are called \textbf{Indicator Variables}
    \end{remark}
\end{definition}



\begin{definition}
    \textbf{Probability Mass Function}:
    An equation which defines the probability the X can take on a value, used for Bernoulli Trials:
    $$P(X=x) = p^x(1-p)^{1-x}$$ where $X$ is the random variable and $x$ is the specific value(1,0)
\end{definition}

\begin{definition}
    \textbf{Bernoulli Distribution}:
    A probabilistic deescription of the successes or failures of a single Bernoulli experiment/trial. Probability of Success(1) in a Bernouli trial is seen as $p$ whereas probability of failure(0) is seen as $1-p$
\end{definition}

\begin{definition}
    \textbf{Mean and Variance of a Bernouli Distribution}:
    Because the values a Bernoulli trial can have(1,0), we can see that the mean $\mu$ can be described as $$E(X) = p$$ and the variance as $$Var[X] = pq=p(1-p)$$
\end{definition}

\subsection*{5.4 Binomial Distributions}

\begin{definition}
    \textbf{Binomial Distribution}:
    Distribution of the number of successes of a fixed amount($n$) Bernoulli trials. For a binomial distribution the probability of individual events can be calculated using the formula:
    $$P(X=x) =_nC_x p^xq^{n-x}$$
\end{definition}

\begin{definition}
    \textbf{Mean and Variance of a Binomial Distribution}:
    Because a Binomial distribution is multiple Bernoulli trials the mean can be found by $$\mu = np$$ and the variance can be found by $$\sigma^2 = np(1-p)$$
\end{definition}
 

\subsection*{Cumulative Distribution Functions(CDF)}

\begin{definition}
    \textbf{Cumulative Distribution Functions(CDF)}:
    Used to calculate the accumulated probability left of an area. Essentially the integral for values before X, made for Continous distributions.The PDF for a continuous distribtuion is found by $$f(x) = \frac{1}{b-a}$$ whereas the CDF is found by 

\end{definition}


\subsection*{Expectations of Discrete Randvars}

\begin{definition}
    \textbf{Expected Value of Binomial Distribution}:
    For a Binomial Distribution, the mean ends up being the expected value in that $$E(X) = \mu = n \cdot p$$ across the different values resulting in:
    $$E(x) = \Sigma x \cdot p(x)$$
\end{definition}

\subsection*{Variance of Expectation}

\begin{definition}
    \textbf{Variance of Expected Value}:
    The variance of the expected value depends on the variance of the distribution, which means that we get the relation $$Var(X) = E[(X-\mu)^2] = E(X^2) - (E(X))^2 = E(X^2) - \mu^2$$ which results in $$\sigma^2 = \Sigma(x-\mu)^2 \cdot p(x)$$
\end{definition}

\end{document}